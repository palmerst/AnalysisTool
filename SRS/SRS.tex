\documentclass[english]{article}

% Imported Packages
%------------------------------------------------------------------------------
\usepackage{amssymb}
\usepackage{amstext}
\usepackage{amsthm}
\usepackage{amsmath}
\usepackage{enumerate}
\usepackage{fancyhdr}
\usepackage[margin=1in]{geometry}
\usepackage{graphicx}
\usepackage{extarrows}
\usepackage{setspace}
\usepackage{babel}
\usepackage[utf8]{inputenc}
\usepackage{verbatim}
\usepackage{tabulary}
\usepackage[pageanchor]{hyperref}
%------------------------------------------------------------------------------

% For Graphs
%------------------------------------------------------------------------------
\usepackage{tikz}
\usepackage{verbatim}
\usetikzlibrary{positioning,arrows,shapes}

% Header and Footer
%------------------------------------------------------------------------------
\pagestyle{plain}
\renewcommand\headrulewidth{0.4pt}
\renewcommand\footrulewidth{0.4pt}
%------------------------------------------------------------------------------

\setlength\parindent{0pt}

% Document
%------------------------------------------------------------------------------
\begin{document}

% Title Page & Table of Contents
%------------------------------------------------------------------------------
\pagenumbering{gobble}
\vspace*{\fill}
{\centering{
\Huge{Compensation Analysis Tool}\\
\bigskip
\LARGE{Software Requirements Specification}\\

\bigskip \bigskip \bigskip
\medskip
\begin{tabular}{l l}
    Ali Reda & 1206754\\
    Steven Palmer & 1434676
\end{tabular}\\

\bigskip \bigskip \bigskip
\today
}
\vspace*{\fill}

\newpage

\tableofcontents

\newpage

\pagenumbering{arabic}
\setcounter{page}{1}
}


\color{black}
\section{Introduction}
\label{sec:introduction}
% Begin Section

\subsection{Purpose}
\label{sub:purpose}
% Begin SubSection	
The purpose of this document is to provide a description of the system requirements for the Compensation Analysis Tool software.  The document includes details about the domain of the project, a listing of identified functional and non-functional requirements, as well as requirements on the development and maintenance process.
% End SubSection

\subsection{Overview}
The information in this document is presented in the following order:
\begin{itemize} 
\item The domain of the project is described in \hyperref[sec:domain]{\S\ref*{sec:domain}}.  This sections provides details about the project objectives, the scope of the project, the involved stakeholders, as well as the constraints, assumptions, and risks that will affect the implementation of the project.

\item The functional requirements of the project are listed in \hyperref[sec:functional_requirements]{\S\ref*{sec:functional_requirements}}.
\item The non-functional requirements of the project are listed in \hyperref[sec:non-functional_requirements]{\S\ref*{sec:non-functional_requirements}}.
\item Requirements on the development and maintenance process are given in \hyperref[sec:req_on_dev_and_maint]{\S\ref*{sec:req_on_dev_and_maint}}.  This includes testing procedures, the implementation order of requirements, and likely changes.
\end{itemize}
\subsection{Definitions, Acronyms, and Abbreviations}
\label{sub:definitions_acronyms_and_abbreviations}
% Begin SubSection
\begin{enumerate}[leftmargin=1cm]
	\item[{\bf FIR:}] functional input requirement
	\item[{\bf FDSR:}] functional data structure requirement
	\item[{\bf FAR:}] functional algorithmic requirement
	\item[{\bf FOR:}] functional output requirement
	\item[{\bf AR:}] appearance requirement
	\item[{\bf SR:}] style requirement
	\item[{\bf EUR:}] ease of use requirement
	\item[{\bf LR:}] learning requirement
	\item[{\bf UPR:}] understandability and politeness requirement
    \item[{\bf ACR:}] accessability requirement
    \item[{\bf SLR:}] speed and latency requirement
    \item[{\bf PAR:}] precision and accuracy requirement
    \item[{\bf RFR:}] robustness or fault-tolerance requirement
    \item[{\bf LONGR:}] longevity requirement
    \item[{\bf EPER:}] expected physical environment requirement
    \item[{\bf MR:}] maintenance requirement
    \item[{\bf MSR:}] supportability requirement
    \item[{\bf MAR:}] adaptability requirement
    \item[{\bf SAR:}] access requirement
    \item[{\bf SPR:}] privacy requirement
    \item[{\bf CCR:}] cultural requirement
    \item[{\bf LCR:}] compliance requirement   


\end{enumerate}
% End SubSection

\section{Domain}
\label{sec:domain}
% Begin Section

\subsection{Objectives}
\label{sub:obj}
The objective of this project is to produce a software tool used to assess and rank the most viable locations to start a career in a particular field. The software will produce a list of locations that offer the best income potential for a particular career or field using a mixture of employment and cost of living statistics. In addition, the software will include an option to consider the possibility of commuting in order to reduce the cost of living.\\\\

The following datasets will be used to develop the application:
\begin{itemize}
  \item Occupational employment statistics published by the United States Department of Labor \cite{ref:empstat} will be used to find location-based average compensation for specific careers
  \item Location affordability statistics published by the United States Office of the Secretary of Transportation \cite{ref:afford} will be used in the assessment of location-based costs of living
  \item Geographic location data from the GeoNames database \cite{ref:geo} will be used to create a graph for computing location/commute distances
\end{itemize}

\color{black}
\subsection{Scope}
\label{sub:scope}
% Begin SubSection
The application will be geographically limited to the United States and will only consider income related factors to arrive at a ranking.  These limitations may be expanded upon in the future, but such enhancements are beyond the current scope of the application. The application will be implemented in Java and will be able to be run on any desktop, laptop, or other device with the ability to execute JAR (Java Archive) files.


\subsection{Stakeholders}
The stakeholders of this project include the team members, the instructor and teaching assistants of the CAS 2XB3 course, and any persons seeking employment who would use the tool.

\medskip


The goals of every stakeholder vary. The goal of the team members is to create a useful application that allows the user who is seeking employment to have a simple and effective experience. The instructor and teaching assistant are meant to guild the team members to build the application to be successful. The relationship between all stakeholders is that they all share a common objective. Which is to have a functional application that helps persons looking for employment. 



% End SubSection

\subsection{Constraints}
\label{sub:constraints}
% Begin SubSection
The following constraints will apply to the development of our application:
\begin{enumerate}
    \item The implementation of the application must include at least one sorting algorithm, at least one searching algorithm, and at least one graph-based algorithm.
	\item All milestones outlined in the CAS 2XB3 project description must be met.
	\item All deliverables must be completed in their final form by the due date given in the CAS 2XB3 project description.
\end{enumerate}
% End SubSection

\subsection{Assumptions and Dependencies}
\label{sub:assumptions_and_dependencies}
% Begin SubSection
The following assumptions were made in the development of our requirements:
\begin{enumerate}
	\item The computer/device used to run the application will have the ability to execute JAR files.
\end{enumerate}
% End SubSection

\subsection{Risks}

The risks that may need to be overcome during the implementation of this project are:

\medskip
\begin{enumerate}

\item Successful integration and utilization of the datasets listed in \hyperref[sub:obj]{\S\ref*{sub:obj}} will need to be achieved in order for the application to function properly.
\item Successful teamwork and scheduling will be imperative in meeting all of the deadlines given in the CAS 2XB3 project description.

\end{enumerate}
% End Section


\section{Functional Requirements}
\label{sec:functional_requirements}

\subsection{Input Requirements}
% Begin Section:
\begin{enumerate}[\bf{FIR}1.]
	\item The application must allow the user to select a career field.
	\begin{enumerate}[leftmargin=1cm]
        \item [{\bf Rationale:}] The user must be able to select a career field in order to perform the analysis.
        \item [{\bf Fit Criterion:}] The user is able to select a career field when running an analysis.
	\end{enumerate}

    \item The application must allow the user to (optionally) select a career title.
	\begin{enumerate}[leftmargin=1cm]
        \item [{\bf Rationale:}] The user must have the option of selecting a career title to allow for a more specific analysis.
        \item [{\bf Fit Criterion:}] The user is able to select a career title when running an analysis.
	\end{enumerate}

    \item The application must allow the user to (optionally) select a location restriction.
	\begin{enumerate}[leftmargin=1cm]
        \item [{\bf Rationale:}] The user must have the option to limit analyses within a radius of a particular city.
        \item [{\bf Fit Criterion:}] The user is able to select a location and radius when running an analysis to limit the search.
	\end{enumerate}

    \item The application must allow the user to (optionally) select a commuting distance.
	\begin{enumerate}[leftmargin=1cm]
        \item [{\bf Rationale:}] The user must have the option to limit analyses within a radius of a particular city.
        \item [{\bf Fit Criterion:}] The user is able to select a career field when running an analysis.
	\end{enumerate}

    \item The application must allow the user to run an analysis after all required/optional inputs are specified.
	\begin{enumerate}[leftmargin=1cm]
        \item [{\bf Rationale:}] The application must be able to perform analyses.
        \item [{\bf Fit Criterion:}] The user is able to run an analysis that transforms input data into output data.
	\end{enumerate}
	

\end{enumerate}

\subsection{Data Structure Requirements}
% Begin Section:
\begin{enumerate}[\bf{FDSR}1.]
	\item The application must load and store employment statistics information using the data available from the U. S. Bureau of Labor Statistics.
	\begin{enumerate}[leftmargin=1cm]
        \item [{\bf Rationale:}] Employment data is required to perform the analyses.
        \item [{\bf Fit Criterion:}] Employment data is successfully loaded/stored into an abstract datatype by the application.
	\end{enumerate}

	\item The application must load and store cost of living information using the data available from the U. S. Office of the Secretary of Transportation.
	\begin{enumerate}[leftmargin=1cm]
        \item [{\bf Rationale:}] Cost of living data is required to perform the analyses.
        \item [{\bf Fit Criterion:}] Cost of living data is successfully loaded/stored into an abstract datatype by the application.
	\end{enumerate}
	
    \item The application must load and store geographic location information for U.S. cities using the data available from the GeoNames database.
	\begin{enumerate}[leftmargin=1cm]
        \item [{\bf Rationale:}] The geographic location data of U.S. cities is required to perform the analyses.
        \item [{\bf Fit Criterion:}] Geographic location data is successfully loaded/stored into an abstract datatype by the application.
	\end{enumerate}
\end{enumerate}
% End Section

\subsection{Algorithmic Requirements}
% Begin Section:

\begin{enumerate}[\bf{FAR}1.]

    \item The application must be able to sort employment data efficiently for use in the analyses.
	\begin{enumerate}[leftmargin=1cm]
        \item [{\bf Rationale:}] Employment data must be sorted to enable efficient searching.
        \item [{\bf Fit Criterion:}] The application successfully sorts location data in $n~log~n$ time.
	\end{enumerate}

	\item The application must be able to search employment data efficiently for use in the analyses.
	\begin{enumerate}[leftmargin=1cm]
        \item [{\bf Rationale:}] Searching of employment data will be performed frequently by the application.
        \item [{\bf Fit Criterion:}] The application searches for an entry in the loaded employment data and returns a result in $log~n$ time or better.
	\end{enumerate}

    \item The application must be able to sort cost of living data efficiently for use in the analyses.
	\begin{enumerate}[leftmargin=1cm]
        \item [{\bf Rationale:}] Cost of living data must be sorted to enable efficient searching.
        \item [{\bf Fit Criterion:}] The application successfully sorts cost of living data in $n~log~n$ time.
	\end{enumerate}

	\item The application must be able to search cost of living data efficiently for use in the analyses.
	\begin{enumerate}[leftmargin=1cm]
        \item [{\bf Rationale:}] Searching of cost of living data will be performed frequently by the application.
        \item [{\bf Fit Criterion:}] The application searches for an entry in the loaded cost of living data and returns a result in $log~n$ time or better.
	\end{enumerate}

    \item The application must be able to sort location data efficiently for use in the analyses.
	\begin{enumerate}[leftmargin=1cm]
        \item [{\bf Rationale:}] Location data must be sorted to enable efficient searching.
        \item [{\bf Fit Criterion:}] The application successfully sorts location data in $n~log~n$ time.
	\end{enumerate}

	\item The application must be able to search location data efficiently for use in the analyses.
	\begin{enumerate}[leftmargin=1cm]
        \item [{\bf Rationale:}] Searching of location data will be performed frequently by the application.
        \item [{\bf Fit Criterion:}] The application searches for an entry in the loaded location data and returns a result in $log~n$ time or better.
	\end{enumerate}

    \item The application must be able to transform a set of inputs into a ranked listing of the locations that yield the highest adjusted compensation (the output).
	\begin{enumerate}[leftmargin=1cm]
        \item [{\bf Rationale:}] This is the main function of the application.  This requirement depends on the previous algorithmic requirements for sorting and searching.
        \item [{\bf Fit Criterion:}] A given input is successfully transformed into an output consisting of a list of locations ranked by highest adjusted compensation.
	\end{enumerate}

	

\end{enumerate}

\subsection{Output Requirements}
% Begin Section:
\begin{enumerate}[\bf{FOR}1.]
    \item The application must print the output data in a way that is readable by the user.
	\begin{enumerate}[leftmargin=1cm]
        \item [{\bf Rationale:}] The user needs to be able to see the result of the analysis.
        \item [{\bf Fit Criterion:}] Output data is formatted and displayed in the application.
	\end{enumerate}
	

\end{enumerate}

\section{Non-Functional Requirements}
\label{sec:non-functional_requirements}
% Begin Section
\subsection{Look and Feel Requirements}
\label{sub:look_and_feel_requirements}
% Begin SubSection

\subsubsection{Appearance Requirements}
\label{ssub:appearance_requirements}
% Begin SubSubSection
\begin{enumerate}[{AR}1. ]

\item The Application should have a default bright and colorful scheme, using uniform colors and a uniform layout.

\item All tabs, buttons and input fields must be easily visible.


\end{enumerate}
% End SubSubSection

\subsubsection{Style Requirements}
\label{ssub:style_requirements}
% Begin SubSubSection
\begin{enumerate}[{SR}1. ]
\item The Application should provide a uniform look and feel between all the pages and tabs.

\item Results returned from the input must be clear and uniform.
\end{enumerate}
% End SubSubSection

% End SubSection

\subsection{Usability and Humanity Requirements}
\label{sub:usability_and_humanity_requirements}
% Begin SubSection

\subsubsection{Ease of Use Requirements}
\label{ssub:ease_of_use_requirements}
% Begin SubSubSection
\begin{enumerate}[{EUR}1. ]

\item Searching through career fields, and finding the needed result must be easy and fast to do.

\item Choosing your current location, and accepted radius must be easy to do.

\end{enumerate}

% End SubSubSection

\subsubsection{Learning Requirements}
\label{ssub:learning_requirements}
% Begin SubSubSection
\begin{enumerate}[{LR}1. ]
\item The product should have an intuitive layout, it will take the user no longer than 10 minutes to learn to use.
\end{enumerate}
% End SubSubSection

\subsubsection{Understandability and Politeness Requirements}
\label{ssub:understandability_and_politeness_requirements}
% Begin SubSubSection
\begin{enumerate}[{UPR}1. ]
\item Application must use proper English, and must be easy to read.

\item No inappropriate language must be included.
\end{enumerate}
% End SubSubSection

\subsubsection{Accessibility Requirements}
\label{ssub:accessibility_requirements}
% Begin SubSubSection
\begin{enumerate}[{ACR}1. ]
\item Font sizes, input boxes, and results should all be appropriate size.
\end{enumerate}
% End SubSubSection

% End SubSection

\subsection{Performance Requirements}
\label{sub:performance_requirements}
% Begin SubSection

\subsubsection{Speed and Latency Requirements}
\label{ssub:speed_and_latency_requirements}
% Begin SubSubSection
\begin{enumerate}[{SLR}1. ]
\item Load times between the user inputting their information to the program giving the results must take no longer than 10 seconds.
\end{enumerate}
% End SubSubSection

\subsubsection{Precision or Accuracy Requirements}
\label{ssub:precision_or_accuracy_requirements}
% Begin SubSubSection
\begin{enumerate}[{PAR}1. ]
\item The application must give appropriate results depending on the exact radius the user requested.

\item Inputted radius must be accurate to $\sim$100 meters.
\end{enumerate}
% End SubSubSection

\subsubsection{Reliability and Availability Requirements}
\label{ssub:reliability_and_availability_requirements}
% Begin SubSubSection
\begin{enumerate}[{RAR}1. ]
\item The system shall be available at all times.
\end{enumerate}
% End SubSubSection

\subsubsection{Robustness or Fault-Tolerance Requirements}
\label{ssub:robustness_or_fault_tolerance_requirements}
% Begin SubSubSection
\begin{enumerate}[{RFR}1. ]
\item The application must give the user a relevant error message when an error occurs.

\item The application must insure the user inputs all necessary fields before searching.

\item The application must save the users previous entries.

\end{enumerate}
% End SubSubSection

\subsubsection{Longevity Requirements}
\label{ssub:longevity_requirements}
% Begin SubSubSection
\begin{enumerate}[{LONGR}1. ]
\item The application must run as long as it is on a compatible OS.
\end{enumerate}
% End SubSubSection

% End SubSection

\subsection{Operational and Environmental Requirements}
\label{sub:operational_and_environmental_requirements}
% Begin SubSection

\subsubsection{Expected Physical Environment Requirements}
\label{ssub:expected_physical_environment}
% Begin SubSubSection
\begin{enumerate}[{EPER}1. ]
\item The application will be used by users looking for potential employment.

\end{enumerate}
% End SubSubSection


% End SubSection

\subsection{Maintainability and Support Requirements}
\label{sub:maintainability_and_support_requirements}
% Begin SubSection

\subsubsection{Maintenance Requirements}
\label{ssub:maintenance_requirements}
% Begin SubSubSection
\begin{enumerate}[{MR}1. ]
\item The system should be documented in such a way that maintaining and updating it is easy for programmers who did not build it initially.

\end{enumerate}
% End SubSubSection

\subsubsection{Supportability Requirements}
\label{ssub:supportability_requirements}
% Begin SubSubSection
\begin{enumerate}[{MSR}1. ]
\item The application must run on windows, and be accessible to anyone with a windows computer.

\end{enumerate}
% End SubSubSection

\subsubsection{Adaptability Requirements}
\label{ssub:adaptability_requirements}
\begin{enumerate}[{MAR}1. ]
\item The system should be built in such a way that adding entirely new features does not require the rewriting of old features.

\end{enumerate}
% End SubSubSection

% End SubSection

\subsection{Security Requirements}
\label{sub:security_requirements}
% Begin SubSection

\subsubsection{Access Requirements}
\label{ssub:access_requirements}
% Begin SubSubSection
\begin{enumerate}[{SAR}1. ]
\item The application will be accessible to users with a compatible computer/device.


\end{enumerate}
% End SubSubSection
\subsubsection{Privacy Requirements}
\label{ssub:privacy_requirements}
\begin{enumerate}[{SPR}1. ]
\item The system shall protect the privacy of users, not allowing users to access others users personal information.
\end{enumerate}
% End SubSubSection

% End SubSection

\subsection{Cultural and Political Requirements}
\label{sub:cultural_and_political_requirements}
% Begin SubSection

\subsubsection{Cultural Requirements}
\label{ssub:cultural_requirements}
% Begin SubSubSection
\begin{enumerate}[{CCR}1. ]
\item The product will use Canadian English spelling.
\end{enumerate}
% End SubSubSection

% End SubSection

\subsection{Legal Requirements}
\label{sub:legal_requirements}
% Begin SubSection

\subsubsection{Compliance Requirements}
\label{ssub:compliance_requirements}
% Begin SubSubSection
\begin{enumerate}[{LCR}1. ]
\item This product will comply with the Personal Information Protection and Electronic Documents Act.
	\item This product will comply with the Ontario Freedom of Information and Privacy Act.
\end{enumerate}
% End SubSubSection
% End SubSubSection

% End SubSection

\section{Requirements on the Development and Maintenance Process}
\label{sec:req_on_dev_and_maint}
\subsection{Quality Control Procedures}
Quality will be controlled by employing both unit testing and manual testing by the developers.  Unit testing will be carried out using JUnit and will include tests that ensure the accuracy of all data structures and algorithms implemented in this project.  Manual testing will include visual inspections to ensure that input and output fields display correctly and function correctly.

\subsection{Implementation of Requirements}
The implementation of requirements will be carried out in the following order:

\begin{enumerate}
\item Data structure and algorithmic functional requirements will be implemented first and will act as the core of the application.
\item Input and output functional requirements will be implemented next to create a fully working application.
\item Non-functional requirements will be checked for adherence.  If some non-functional requirements are not met, small changes to the implementation should be sufficient to remedy them.
\end{enumerate}

\subsection{Likely Changes}
At this point, there are no likely changes to the functional or non-functional requirements described in this document, or to the development and maintenance process.
% End SubSection


\newpage
\begin{thebibliography}{99} % Bibliography - this is intentionally simple in this template

\bibitem[1]{ref:empstat}
U.S. Department of Labor, Bureau of Labor Statistics.
\newblock (2015, Nov 4).
\newblock {\em Occupational Employment Statistics - Employment and Wages} [Online].
\newblock Available: \url{http://catalog.data.gov/dataset/occupational-employment-statistics-employment-and-wages}

\bibitem[2]{ref:afford}
Office of the Secretary of Transportation.
\newblock (2015, Aug 28).
\newblock {\em Location Affordability Index: All Census Places} [Online].
\newblock Available: \url{http://catalog.data.gov/dataset/location-affordability-index-all-census-places}

\bibitem[3]{ref:geo}
GeoNames.
\newblock (2016, Feb 7).
\newblock {\em GeoNames} [Online].
\newblock Available: \url{http://download.geonames.org/export/dump/}

\end{thebibliography}

\end{document}
%------------------------------------------------------------------------------
